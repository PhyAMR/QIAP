\documentclass[10pt]{article}
\usepackage[utf8]{inputenc}
\usepackage[T1]{fontenc}
\usepackage{graphicx}
\usepackage[export]{adjustbox}
\graphicspath{ {./images/} }
\usepackage{amsmath}
\usepackage{amsfonts}
\usepackage{amssymb}
\usepackage[version=4]{mhchem}
\usepackage{stmaryrd}
\usepackage{bbold}

\begin{document}
\begin{enumerate}
  \setcounter{enumi}{3}
  \item Perturbation Theory (the indefendent)\\
done in a way that is useful\\
\includegraphics[max width=\textwidth, center]{2025_10_16_f6b2ddb567eefef2c7a2g-1(1)}
\end{enumerate}

UNPERTURBED

\section*{hamiltonian}
$H_{0}\left|\varepsilon^{(0)}, J\right\rangle=\varepsilon^{(0)}\left|\varepsilon^{(0)}, J\right\rangle$\\
\includegraphics[max width=\textwidth, center]{2025_10_16_f6b2ddb567eefef2c7a2g-1}

EXAMPLE

$$
\left(\begin{array}{rll}
\varepsilon_{0} & & \\
& \varepsilon_{0} & \\
& & \rightarrow \\
& & \mid \varepsilon_{1}
\end{array}\right) \xrightarrow{\rightarrow}\left|\varepsilon_{0}, 1\right\rangle=\left(\begin{array}{l}
1 \\
0 \\
0
\end{array}\right)=\left(\begin{array}{l}
0 \\
1 \\
0
\end{array}\right)=\left(\begin{array}{l}
0 \\
0 \\
1
\end{array}\right)
$$

$\widetilde{\varepsilon}_{n}=\varepsilon_{n}^{(0)}+\lambda \varepsilon_{n}^{(1)}+\lambda^{2} \varepsilon_{n}^{(2)}+\ldots$\\
$\left|\tilde{\varepsilon}_{n}\right\rangle=\left|\varepsilon_{n}^{(0)}\right\rangle+\lambda\left|\varepsilon_{n}^{(1)}\right\rangle+\cdots$ Also WITH $J$\\
$\left(H_{0}+\lambda V\right)\left(\left|\varepsilon^{(0)}, J\right\rangle+\lambda\left|\varepsilon^{(1)}, J^{1}\right\rangle+\ldots\right)=\left(\varepsilon^{(0)}+\lambda \varepsilon^{(1)}+\ldots\right)\left(\left|\varepsilon^{(1)}, J\right\rangle+\lambda \mid \varepsilon^{(1)}, J\right.$\\
ORDER $1=\lambda^{\circ}$\\
$H_{0}\left|\varepsilon^{(0)}, J\right\rangle=\varepsilon^{(0)}\left|\varepsilon^{(0)}, J\right\rangle$ well, AT LEAST IT IS CONSISTENT

ORSER $\lambda=\lambda^{1}$\\
$V\left|\varepsilon^{(0)}, J\right\rangle+H_{0}\left|\varepsilon^{(1)}, J^{\prime}\right\rangle=\varepsilon^{(1)}\left|\varepsilon^{(0)}, J\right\rangle+\varepsilon^{(0)}\left|\varepsilon^{(1)}, J^{\prime}\right\rangle$

$$
\Pi=\Pi^{+}=\Pi^{2}
$$

$\left.\begin{array}{l}\text { I now DEFINE THE PROSECYOR } \\ \text { ONTO THE } \varepsilon^{(0)} \text { EIGENSPACE OF } H_{0}\end{array}\right\} \quad \Pi_{\varepsilon_{0}}=\sum_{\zeta}\left|\varepsilon^{(0)}, J\right\rangle\left\langle\varepsilon^{(0)}, J\right|$

$$
\prod_{\varepsilon_{0}} H_{0}=H \prod_{0}=\varepsilon^{(0)} \Pi \quad \ldots \text { AND I HULTIPLY LEFT. }
$$

$\Pi_{\varepsilon_{0}} V\left|\varepsilon^{(0)}, J\right\rangle+\underbrace{}_{\varepsilon_{0}} H\left|\varepsilon^{(1)}, J^{\prime}\right\rangle=\varepsilon^{(1)} \Pi_{\varepsilon_{0}}\left|\varepsilon^{(0)}, J\right\rangle+\varepsilon^{(0)} \Pi_{\varepsilon_{0}}\left|\varepsilon^{(0)}, J^{\prime}\right\rangle$

$$
\varepsilon^{(0)} \Pi_{\varepsilon^{(0)}}\left|\varepsilon^{(+)}, J^{\prime}\right\rangle
$$

$\left|\varepsilon^{(0)}, J\right\rangle=\pi_{\varepsilon^{(0)}}\left|\varepsilon^{(0)}, J\right\rangle$

$$
\prod_{\varepsilon(0)}\left|\varepsilon^{(0)}, j\right\rangle=\left|\varepsilon^{(0)}, j\right\rangle
$$

$$
(\underbrace{\Pi_{\varepsilon_{0}} \vee \Pi_{\varepsilon_{0}}})\left|\varepsilon^{(0)}, J\right\rangle=\varepsilon^{(1)}\left|\varepsilon^{(0)}, J\right\rangle \quad \forall J
$$

$\Rightarrow \underbrace{(\pi V \pi)^{+}=\pi^{+} V^{+} \pi^{+}=\pi V \pi}_{\text {HERMITIAN }}$\\
Eigenvalue Equation\\
$\rightarrow$ IT TEUS US HOW THE SEGENERACY IS REMOVES AND\\
HOW THE RESOLVED STATES LOOK LIKE\\
Notice $\rightarrow$ the resolves states are (NOT) a correction from an arbitrary $|\varepsilon, j\rangle$ (see example later)

EXCERCISE\\
$\lambda H_{\varepsilon_{i}^{(1)}}^{(1)}\left|\varepsilon^{(0)}, J\right\rangle=\lambda \varepsilon^{(1)}\left|\varepsilon^{(0)}, J\right\rangle$\\
where $\lambda M_{\varepsilon_{0}}^{(1)}=\lambda\left(\Pi_{\varepsilon^{(0)}} \vee \Pi_{\varepsilon^{(0)}}\right)$

CONTRACT * WITH

$$
\left\langle\varepsilon_{n}^{(0)}, J\right| \text { AND }
$$

LEARN SOMETHING ABOUT $\left|\varepsilon^{(1)}, J\right\rangle$

GA HIGHER ORDERS OF DEG-REMOVING HAMICTONIANS $L$ USUALLY THE COWEST NONZERO ORDER COUNTS

NON-DEG

$$
\begin{aligned}
& H_{0}^{(1)}=\Pi_{\varepsilon_{0}} V \Pi_{\varepsilon_{0}} \\
& \varepsilon^{(1)}=\left\langle\varepsilon_{0}\right| V\left|\varepsilon_{0}\right\rangle \\
& H_{\varepsilon_{0}}^{(2)}=\Pi_{\varepsilon_{0}} \vee R_{\varepsilon_{0}} \vee \Pi_{\varepsilon_{0}} \\
& \Leftrightarrow \varepsilon^{(2)}=\sum_{\varepsilon_{n} \neq \varepsilon_{0}} \frac{\left.\left|\left\langle\varepsilon_{n}\right| V\right| \varepsilon_{0}\right\rangle\left.\right|^{2}}{\varepsilon_{n}^{(0)}-\varepsilon_{n}^{(0)}} \\
& =\left\langle\varepsilon_{0}\right| V \underbrace{\left(\sum_{\varepsilon_{n} \varepsilon_{0}} \frac{\left|\varepsilon_{n} \times \varepsilon_{n}\right|}{\varepsilon_{0}^{(0)}-\varepsilon_{n}^{(0)}}\right)} V\left|\varepsilon_{0}\right\rangle \\
& \text { MOORE-PENROSE } \\
& \text { PSEUDOINVERSE } \\
& \text { (INVERY ONCY THE) } \\
& \downarrow \\
& R_{\varepsilon_{0}}=\left(\begin{array}{ccl}
0 & & \\
\frac{1}{\varepsilon_{0}-\varepsilon_{1}} & \\
& \frac{1}{\varepsilon_{0}-\varepsilon_{i}}
\end{array}\right) \\
& A=\left(\begin{array}{lll}
0 & & \\
& 1 & \\
& & 2
\end{array}\right) \quad A^{\prime \prime \prime}=\left(\begin{array}{lll}
0 & & \\
& 1 & \\
& & 1 / 2
\end{array}\right) \\
& \varepsilon^{(2)}=\left\langle\varepsilon_{0}\right| V\left(\varepsilon_{0}^{(0)} \mathbb{1}-H_{0}\right)^{-1} V \mid \varepsilon \\
& H_{\varepsilon_{0}}^{(3)}=\Pi_{\varepsilon_{0}} V R_{\varepsilon_{0}} V R_{\varepsilon_{0}} V \Pi_{\varepsilon_{0}}-\Pi_{\varepsilon_{0}} V \Pi_{\varepsilon_{0}} V R_{\varepsilon_{0}}^{2} V \Pi_{\varepsilon_{0}} \\
& H_{\varepsilon_{0}}^{(4)}=\Pi_{\varepsilon_{0}} V R_{\varepsilon_{0}} V R_{\varepsilon_{0}} V R_{\varepsilon_{0}} V \Pi_{\varepsilon_{0}}-\Pi_{\varepsilon_{0}} V R_{\varepsilon_{0}}^{2} V \Pi_{\varepsilon_{0}} V R_{\varepsilon_{0}} V \Pi_{\varepsilon_{0}} \\
& -\pi \vee \pi \vee R \vee R^{2} \vee \pi-\pi \vee \pi \vee R^{2} \vee R V \pi \\
& +\pi V \pi V \pi V R^{3} V \pi
\end{aligned}
$$

(5) nonsense

EXERCISE THE LAMBDA SYSTEM (USEEVI FOR

$$
\frac{\hbar}{|0\rangle} \frac{|e\rangle}{|1\rangle}
$$

$$
|0\rangle=\left(\begin{array}{l}
1 \\
0 \\
0
\end{array}\right) \quad|e\rangle=\left(\begin{array}{l}
0 \\
1 \\
0
\end{array}\right) \quad|1\rangle=\left(\begin{array}{l}
0 \\
0 \\
1
\end{array}\right)
$$

$$
H_{0}=\left(\begin{array}{ccc}
0 & & \\
& +\Delta & \\
& & 0
\end{array}\right)
$$

$$
V=\left(\begin{array}{ccc}
0 & \Omega & \\
\Omega & 0 & \Omega \\
& \Omega & 0
\end{array}\right)=\Omega\left(\begin{array}{ll}
1 & 1 \\
1 & 1
\end{array}\right)
$$

UN PERTURBED\\
GAMILTONIAN

$$
\left.\left.\begin{array}{lll}
\Pi_{0}=\left(\begin{array}{ll}
1 & \\
0 & 1 \\
& 1
\end{array}\right) & R_{0}=\left(\begin{array}{cc}
0 & \\
-\frac{1}{\Delta} & \\
& \\
\Pi_{\Delta}=\left(\begin{array}{ll}
0 & \\
1
\end{array}\right)
\end{array}\right. & H_{0}^{(1)}=0 \\
& 0
\end{array}\right)\left(\begin{array}{cc}
\frac{1}{\Delta} & \\
& 0 \\
& \\
& +\frac{1}{\Delta}
\end{array}\right)\right)\left(H_{\Delta}^{(1)}=0\right.
$$

$H_{0}^{(2)}=\Pi_{0} \vee R_{0} \vee \Pi_{0}=\frac{\Omega^{2}}{\Delta}\left(\begin{array}{ll}1 & 0 \\ 1 & 1\end{array}\right)\left(\begin{array}{ll}0 & 1 \\ 1 & 1 \\ 1 & 0\end{array}\right)\left(\begin{array}{ll}0 & \\ 1 & 0\end{array}\right)\left(\begin{array}{ll}1 & 1 \\ 1 & 1\end{array}\right)\left(\begin{array}{ll}1 & \\ 0 & 1\end{array}\right)=$

$$
\Delta\left(t+\frac{2 \Omega^{2}}{\Delta^{2}}\right)=-\frac{\Omega^{2}}{\Delta}\left(\begin{array}{ll}
1 & 1 \\
1 & 0 \\
1 & 1
\end{array}\right)
$$

\begin{center}
\includegraphics[max width=\textwidth]{2025_10_16_f6b2ddb567eefef2c7a2g-4}
\end{center}

$$
\begin{aligned}
& \text { DARH } \\
& \text { STAT } \\
& \left.\frac{2 \Omega^{2}}{\Delta}\right)
\end{aligned}
$$

$$
\begin{aligned}
& \frac{|0\rangle+|1\rangle}{\sqrt{2}} \leadsto \varepsilon^{(2)}=-\frac{2 \Omega^{2}}{\Delta} \\
& \frac{|0\rangle-|1\rangle}{\sqrt{2}} \leadsto \varepsilon^{(2)}=0 \text { DARK }
\end{aligned}
$$

$H_{2}^{(2)}=\ldots .=|e \times e|\left(+\frac{2 \Omega^{2}}{\Delta}\right) \quad \varepsilon_{e}^{(0)}+\varepsilon_{e}^{(2)}=\Delta+\frac{2 \Omega^{2}}{\Delta}$\\
11)\\
\includegraphics[max width=\textwidth, center]{2025_10_16_f6b2ddb567eefef2c7a2g-4(1)}

\section*{FUL RABI}
$$
=\Delta\left(1+\frac{2 \Omega^{2}}{\Delta^{2}}\right)
$$

frequency $\left|\left(0-\frac{2 \Omega^{2}}{\Delta}\right)\right|=\frac{2 \Omega^{2}}{\Delta}=\Delta\left(\frac{2 \Omega^{2}}{\Delta^{2}}\right) \ll \Delta$\\
\includegraphics[max width=\textwidth, center]{2025_10_16_f6b2ddb567eefef2c7a2g-4(2)}

\section*{SAME Problem BUT (NO) PERTURBATION tHEORY}
$\hbar=1$

$$
\begin{array}{ll}
H_{0}=\left(\begin{array}{ccc}
0 & & \\
& +\Delta & \\
& & 0
\end{array}\right) \quad V=\left(\begin{array}{ccc}
0 & \Omega & \\
\Omega & 0 & \Omega \\
& \Omega & 0
\end{array}\right) & \text { EXACT } \\
H_{\text {TOT }}=H_{0}+V=\Delta\left(\begin{array}{ccc}
0 & \eta & \\
\eta & 1 & \eta \\
\eta & 0
\end{array}\right) \quad \text { WITH } \quad \eta=\frac{\Omega}{\Delta} \quad \begin{array}{c}
\text { SHALU } \\
\text { PARAMETER }
\end{array}
\end{array}
$$

$\left|\begin{array}{ccc}-\lambda & \eta & \\ \eta & 1-\lambda & \eta \\ & \eta & -\lambda\end{array}\right|=P(\lambda)=\lambda^{2}(1-\lambda)+2 \lambda \eta^{2}=-\lambda\left(\lambda^{2}-\lambda-2 \eta^{2}\right)$

$$
P(\lambda)=0 \rightarrow \lambda=\frac{1}{2}\left(1 \pm \sqrt{1+8 \eta^{2}}\right.
$$

$P(\lambda)=0 \rightarrow \lambda={ }_{0}$\\
$\Delta \sim \frac{1}{\uparrow} \Delta\left(\frac{1}{2}+\frac{1}{2} \sqrt{1+8 \frac{\Omega^{2}}{\Delta^{2}}}\right) \geqslant \Delta\left(\frac{1}{2}+\frac{1}{2}\left(1+4 \frac{\Omega^{2}}{\Delta^{2}}\right)\right)$\\
\includegraphics[max width=\textwidth, center]{2025_10_16_f6b2ddb567eefef2c7a2g-5(1)}\\
\includegraphics[max width=\textwidth, center]{2025_10_16_f6b2ddb567eefef2c7a2g-5}

Secular Motion

Microhotion $\mu^{\mu^{m} \eta_{\eta}}$\\
see marco's lecture\\
$\sim$ LIKE $A$ RABI $\sigma^{*}$\\
$|0\rangle{ }^{\curvearrowleft}|1\rangle$\\
EREQUENCY $\left|0-\frac{2 \Omega^{2}}{\Delta}\right|=\frac{2 \Omega^{2}}{\Delta}$\\
DERIOD $\frac{\pi \Delta}{\Omega^{2}}$ (slow)\\
WIGGLES OF AOPULATION OF $|e\rangle$

FREQUENCY $\sim \Delta$\\
PERIOD $\frac{2 \pi}{\Delta}$ (FAST)\\
5. Time-Orderes Exponential

$$
\begin{aligned}
i \frac{d}{d t}|\psi(t)\rangle & =H(t)|\psi(t)\rangle \\
i \frac{d}{d t} U\left(t, t_{0}\right)\left|\psi \phi_{0}\right\rangle & =H(t) U\left(t, t_{0}\right)\left|\psi_{0}\right\rangle
\end{aligned}
$$

$|\psi(t)\rangle=U\left(t, t_{0}\right)\left|\psi_{0}\right\rangle \quad$ WITH

$$
\underset{\substack{\text { ASSITIVITY } \\ U\left(t_{2}, t_{1}\right) \cup\left(t_{1}, t_{0}\right)}}{=U\left(t_{2}, t_{0}\right)}
$$

$$
U(t, t)=\mathbb{1} \quad \begin{gathered}
\text { Ideourity } \\
\text { CONNETION }
\end{gathered}
$$

SHAU DREHENT $\rangle \delta t \leadsto \quad U(t+\delta t, \underbrace{\left.t_{0}\right)}_{\text {TAYCOR }}=U\left(t, t_{0}\right)+\delta t \frac{d U}{d t}\left(t, t_{0}\right)+\theta\left(\delta t^{2}\right)$

$$
\begin{gathered}
U(t+\delta t) U\left(t, t_{0}\right)=U\left(t+\delta t, t_{0}\right)=(1-i \delta H(t)) U\left(t, t_{0}\right)+\theta\left(\delta t^{2}\right) \\
U(t+\delta t, t) \approx \exp (-i H(t) \delta t)+\theta\left(\delta t^{2}\right)
\end{gathered}
$$

$t_{0} \quad$ sivide in $N$ wifelenals $t>t_{0}$

$$
\delta t=\frac{t-t_{0}}{N}
$$

$U\left(t, t_{0}\right)=U(t, t-\delta t) \frac{U(t-\delta t, t-2 \delta t) \cdots C}{\text { THE ORSER is THPORTANT }}\left(t_{0}+\delta t, t_{0}\right)$

$$
=\exp (-i \delta t H(t-\delta t)) \exp (-i \delta t H(t-2 \delta t)) \cdots \exp (-i \delta t H(t))+\theta
$$

$$
U\left(t, t_{0}\right)=\lim _{\delta t \rightarrow 0}\left(\sum_{\text {EXPRESSION }}^{T H 1 S}(V)=\operatorname{Iexp}\left(-i \int_{t_{0}}^{t} H\left(t^{\prime}\right) d t^{\prime}\right)\right.
$$

(or $N \rightarrow \infty$ )\\
THE HISTORICAL REASON WHY IT IS factoral! WRITMEN LIKE THIS IS DUE TO THE\\
Also-ReE $\downarrow{ }_{t}$ DYSON SERIES\\
$U\left(t, t_{0}\right)=1+\sum_{n=1}^{\infty}(-2)^{n} \int_{t_{0}}^{t} d t_{1} \int_{t_{0}}^{t_{1}} d t_{2} \ldots \int_{t_{0}}^{t_{n}-t_{n}} d t_{n} H\left(t_{1}\right) H\left(t_{2}\right) \ldots H\left(t_{n}\right)$


\end{document}