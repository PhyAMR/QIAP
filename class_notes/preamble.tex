% This file contains the shared preamble for your book.

% FONT and ENCODING
\usepackage[T1]{fontenc}
\usepackage[utf8]{inputenc}
\usepackage[spanish, es-tabla]{babel}

% MATH
\usepackage[mathscr]{euscript}
\usepackage{amsmath, amssymb}
\usepackage{physics}
\usepackage{nccmath}
\usepackage{cancel}

% GRAPHICS & FIGURES
\usepackage{graphicx}
\usepackage{subcaption} % Recommended over subfig for kaobook
\setkeys{Gin}{width=\linewidth,totalheight=\textheight,keepaspectratio}
\graphicspath{{graphics/}}

% TABLES
\usepackage{array, multirow, multicol}

% TIKZ and DIAGRAMS
\usepackage{tikz}
\usetikzlibrary{calc, arrows.meta, chains, positioning, decorations.text}
\usepackage{venndiagram}

% THEOREMS and BOXES
\usepackage{thmtools}
\usepackage[framemethod=TikZ]{mdframed}
\usepackage{tcolorbox}
\tcbuselibrary{most}

% OTHER PACKAGES
\usepackage{float}
\usepackage[footnote]{witharrows}
\usepackage{svrsymbols}
\usepackage{fontawesome}

% --- CUSTOM COMMANDS AND ENVIRONMENTS (from your original file) ---

% THEOREM-LIKE ENVIRONMENTS
\declaretheorem[thmbox=L]{Hipótesis}
\declaretheorem[thmbox=L]{Aproximación}
\declaretheorem[thmbox=M]{Postulado}
\declaretheorem[thmbox=M]{Principio}
\declaretheorem[thmbox=M]{Ley}
\declaretheorem[thmbox=S]{Teorema}

% DEMONSTRATION ENVIRONMENT (mdframed)
\newcounter{dem}[chapter]\setcounter{dem}{0}
\renewcommand{\thedem}{\Roman{dem}}
\newenvironment{dem}[2][]{%
\refstepcounter{dem}%
\ifstrempty{#1}% 
{\mdfsetup{% 
frametitle={% 
\tikz[baseline=(current bounding box.east),outer sep=0pt]
\node[anchor=east,rectangle,fill=white]
{\strut Demostración~\thedem};
}}% 
}
{\mdfsetup{% 
frametitle={% 
\tikz[baseline=(current bounding box.east),outer sep=0pt]
\node[anchor=east,rectangle,fill=white]
{\begin{minipage}{0.99\linewidth}Demostración~\thedem~#1\end{minipage}};}}% 
}
\mdfsetup{innertopmargin=10pt,linecolor=black,% 
linewidth=0.5pt,topline=true,% 
frametitleaboveskip=\dimexpr-\ht\strutbox\relax
}
\begin{mdframed}[]\relax%
\label{#2}}{\end{mdframed}}

% EXERCISE ENVIRONMENT (tcolorbox)
\newtcolorbox[auto counter,
              number within=chapter,
              list inside=ej
              ]{ej}[1][]{
    enhanced, breakable,
    title={{\begin{minipage}{\linewidth}\textbf{Ejercicio}~\thetcbcounter.~\textit{#1}\end{minipage}}},
    halign title=left,
    sharp corners,
    colback=white,
    coltitle=black,
    colbacktitle=white,
    boxrule=0pt,frame hidden,
    underlay unbroken and first={
         \ifnumequal{\tcbsegmentstate}{0}{ 
            \draw[black,double] (interior.north west)--(interior.south west);
        }{\ifnumequal{\tcbsegmentstate}{1}{
                \draw[black,double] (interior.north west)--(segmentation.west);
            \begin{tcbclipinterior}
                    \draw[help lines, step=2.1mm, black!10!white](segmentation.south west) grid (frame.south east);
            \end{tcbclipinterior}
        }{\ifnumequal{\tcbsegmentstate}{2}{
            \begin{tcbclipinterior}
                    \draw[help lines, step=2.1mm, black!10!white](interior.north west) grid (interior.south east);
            \end{tcbclipinterior}
        }}}
    },
    underlay middle and last={
         \ifnumequal{\tcbsegmentstate}{0}{
            \draw[black,double] (interior.north west)--(interior.south west);
        }{\ifnumequal{\tcbsegmentstate}{1}{
                \draw[black,double] (interior.north west)--(segmentation.west);
            \begin{tcbclipinterior}
                    \draw[help lines, step=2.1mm, black!10!white](segmentation.south west) grid (frame.south east);
            \end{tcbclipinterior}
        }{\ifnumequal{\tcbsegmentstate}{2}{
            \begin{tcbclipinterior}
                    \draw[help lines, step=2.1mm, black!10!white](interior.north west) grid (interior.south east);
            \end{tcbclipinterior}
        }}}
    },
    boxed title style={
      colframe=white,
      boxrule=0pt,
      colback=white,
      left=0pt,
      right=0pt},
    attach boxed title to top left={xshift={-5pt}},
    lower separated=false,
    before lower = {\tcbsubtitle[colback=white, opacityback=0, colframe=black, opacityframe=0, boxrule=1pt, height=1cm,  width=2.55cm, valign=center]{\textbf{Solución:}}}
}

% DEFINITION ENVIRONMENT (tcolorbox)
\newtcolorbox[auto counter,
              number within=chapter,
              list inside=defi
              ]{defi}[1][]{
    enhanced,
    title={{\begin{minipage}{0.99\linewidth}\textbf{\textit{#1}}\end{minipage}}},
    halign title=left,
    sharp corners,
    colback=white,
    coltitle=black,
    colbacktitle=white,
    boxrule=0pt,frame hidden,
    overlay unbroken={
      \draw[black,double] (interior.north west)--(interior.south west);
      },
    boxed title style={
      colframe=white,
      boxrule=0pt,
      colback=white,
      left=0pt,
      right=0pt},
    attach boxed title to top left={xshift={-5pt}},
}

% FONT AWESOME SYMBOLS
\def\faChecked{\FA\symbol{"F00C}}
\def\faCrossed{\FA\symbol{"F00D}}

% COURSE-SPECIFIC COMMANDS
\newcommand{\CourseSemester}{ - }

% --- LECTURE LIST SETUP (for kaobook) ---
% This replaces the tocloft setup from your original file.
\newcommand{\listlecturename}{Clases}
\DeclareNewTOC[
    type=lecture,
    name=\listlecturename,
    listname={Índice de clases}
]{lec}
\newcounter{lecnum}
\newcommand{\lecture}[3]{% usage: \lecture{number}{title}{date}
    \clearpage
    \pagestyle{myheadings}
    \setcounter{lecnum}{#1}
    \setcounter{page}{1}
    \refstepcounter{lecture}
    \addcontentsline{lec}{lecture}{Clase~#1:~#2}
    
    % Encabezado de la clase
    \noindent
    \fbox{%
      \begin{minipage}{\dimexpr\textwidth-2\fboxsep-2\fboxrule\relax}
        \centering
        \vspace{2mm}
        {\bfseries #2 \hfill \CourseSemester}\\[4mm]
        {\Large \textbf{Clase #1}}\\[4mm]
        {\normalfont Fecha: #3 \hfill Profesor: \CourseInstructor}\\[2mm]
      \end{minipage}%
    }
    
    \markboth{Clase #1: #2}{Clase #1: #2}
}

% --- PACKAGES THAT MAY CONFLICT WITH KAOBOOK ---
% The 'titlesec' and 'titoc' packages are incompatible with KOMA-Script classes like kaobook.
% The 'geometry' package can interfere with kaobook's layout.
% The 'tocloft' package is not needed as KOMA-Script provides \DeclareNewTOC.
