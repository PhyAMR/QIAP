% Lecture file created by Gemini
% Class: Quantum Information With Atoms and Photons
% Professor: Pietro Silvi
% Date: 2025-10-16
\lecture{1}{Quantum Mech Refresher}{2025-10-16}

% --- Start writing here ---
\captionsetup{singlelinecheck=false}
(0.) Being Sloppy with the constants

SOMETIMES\
(but not always)\
WE MAY "DROP" $\hbar$ OR $C$ OR $K_{B}$

OR WE SAY\
$\\\hbar=1, C=1, K_{B}=1$\
$\begin{array}{ccc}\uparrow & \uparrow & \uparrow \\ \text { PLANCK } & \text { SPEES } & \text { BOCTZMANN } \\ \text { CONSTANT } & \text { OF LIGHT } & \text { CONSTANT }\end{array}$\\
that is because they are not actual constants: they set a conversion rate of a unit to another

$$ 
\begin{aligned}
& c \simeq 3 \cdot 10^{8} \frac{m^{\leftarrow}}{s^{\leftarrow}} \overbrace{\substack{\text { UNIT OF } \\ \text { TENGTH } \\ \text { THE }}}^{\text {UNIT }}

& \text { Allows us to EXPRESS } \
& \text { LENGTGS IN SECONDS } \
& 1 \text { LEGCHS }=3.10^{8} \mathrm{~m} \\
& 47 \times \begin{array}{c}
\text { GARTH } \\ \text { RASII } \end{array} \\
& 0.78 \times \begin{array}{c}
\text { EARTH-HOON } \\ \text { SISTANE } \end{array} \\
& \hbar=(1.055) \cdot 10^{-34} \mathrm{Kg} \cdot \mathrm{~m}^{2} \cdot \mathrm{~s}^{-1} \\
& \text { AUOWS US TO EXPRESS } \
& \text { HASSES IN SECONDS } \
& 1 \text { meter }=\frac{1 / c}{3.3 \cdot 10^{-9}} \text { seconds } \\
& \frac{\hbar}{c^{2}}=1.17 \cdot 10^{-51} \frac{\mathrm{Kg}}{\mathrm{rad} / \mathrm{s}} \\
& 1 \text { "MASS" }_{\text {SECOND }}=1.17 \cdot 10^{-51} \mathrm{Kg} \\
& \xrightarrow[\text { ENERGY }]{\text { REGEST MASS }}\\
& \hbar \omega=\widetilde{m c^{2}} \\
& \hat{\mathrm{cod} / \mathrm{s}} \\
& \underset{\text { MASS }}{\text { ELECTRON }} \quad 9,11 \cdot 10^{-31} \mathrm{Kg} \\
& =7,79 \cdot 10^{20} \mathrm{rad} / \mathrm{s} \\
& (2 \pi \mathrm{~Hz}) \\
& =1.24 \cdot 10^{20} \mathrm{Mz} \text { B1E!? } \\
& \left.\begin{array}{l}
\text { COMPARE TO } \\ \text { VISIBLE LIGHT } \end{array}\right\} \quad 4 \sim 7.9 \cdot 10^{14} \mathrm{~Hz}
\end{aligned}
$$ 

$K_{B}=1.38 \cdot 10^{-23} \frac{m^{2} \mathrm{Kg}}{s^{2} \mathrm{~K}}$\
EXCERCISE. WHAT'S THE DIRECT CONVERTION RATE KELUIN-(SECOND) P?

\begin{itemize}
  \item What's the water melting temperature in (seconss)'
\end{itemize}

Careful >> not all constants are converslons, some carry acyual information

\begin{center}
\begin{tabular}{c|l}
\begin{tabular}{c}
CONVERSION \\ CONSTANTS \\
\end{tabular} & \begin{tabular}{c}
TRUE \\ CONSTANT(S) \\
\end{tabular} \\
\hline
$C$ &  \\
$\hbar$ &  \\
$K_{B}$ & $\alpha=\frac{1}{4 \pi \varepsilon_{0}} \frac{e^{2}}{\hbar c} \approx \frac{1}{137}$ \\
\begin{tabular}{c}
JUST A \\ NUMBER \\
\end{tabular} &  \\
\end{tabular}
\end{center}

\begin{center}
\includegraphics[width=0.5\textwidth]{2025_10_16_f02af6fa434c9f0bcc00g-02}
\end{center}

THERE is A specific ratio between\
$\rightarrow$ ELECTROSTATIC ENERGY\
$\rightarrow$ REST MASS ENERGY\
the ratio is a function of\
$\alpha$\
$\left.\begin{array}{l}\text { DIFFERENT } \\ \hbar, c, K_{B}\end{array}\right\} \begin{aligned} & \text { SAME UNIVERSE } \\ & \text { DIFFERENT UNITS }\end{aligned}$\
$\left.\begin{array}{c}\text { DIFFERENT } \\ \alpha\end{array}\right\}$ ANOTHER UNIVERSE, SIFFERENY FROM OURS

\section*{Revaew: Quantum Mech}
(1) The quantum State $\rightarrow$ (I⿴囗 POSSIBEE) TO SEFANG MEERIMINGT CALEY EVERY MGSURAALE PROPERTY OT A SYSTEM

\begin{center}
\begin{tabular}{|l|l|l|l|l|}
\hline
TAKE $\dot{x}$\operatorname{POSITION} & \includegraphics[width=0.5\textwidth]{2025_10_16_f02af6fa434c9f0bcc00g-03} & IF ONE IS COAPUETELY DETERMINES & $x=x_{0}$ & MSTRIBUTION $\delta\left(x-x_{0}\right)$ \\
\hline
 &  & THE OITHER ONE IS COMPEETECY UNDE TERHIVED &  & DISTRIBUTION "constant" \\
\hline
\end{tabular}
\end{center}

\begin{figure}[h]
\begin{center}
  \includegraphics[width=0.5\textwidth]{2025_10_16_f02af6fa434c9f0bcc00g-03(2)}
\captionsetup{labelformat=empty}
\caption{A SPECIAL CLASS OF STATES}
\end{center}
\end{figure}

$$ \stackrel{\text { PURE }}}{\text { STATES }} \longleftrightarrow \stackrel{\text { A.K.A }}{\longleftrightarrow} \stackrel{\text { VECTOR }}{\longleftrightarrow} \frac{\text { A.K.A }}{\text { STATES }} \longleftrightarrow \text { WAVEFUNCYIONS) } $$

are "the most deterministic" states: you can not add information WITHOUY VIOLATING SOMETHING\
$|\psi\rangle$\
\includegraphics[width=0.5\textwidth, center]{2025_10_16_f02af6fa434c9f0bcc00g-03(1)}\
vectors of a vector space if $\left\{\begin{array}{l}|\psi\rangle+|\varphi\rangle \in \mathbb{H} \\ \lambda|\psi\rangle \in \mathbb{H}\end{array}\right.$
\rightarrow ON COMPLEX FIELD $\lambda|\psi\rangle, \lambda \in \mathbb{C}$
\rightarrow WITH A PROSUCT SCALAR HETRIC $\langle\psi \mid \varphi\rangle$
\rightarrow (CATCH) $|"psi">$ and $\lambda|"psi">$ are actoally the $\frac{\text { Same }}{\text { STATE }}$

$$ \left(\text { RUT } \lambda_{1}|\psi\rangle+\lambda_{2}|\varphi\rangle \text { ANS } \lambda_{2}|\psi\rangle+\lambda_{1}|\varphi\rangle \text { ARE NOT }\right) $$

$\mathbb{1A}$\
the Hilbert metric
$\left.\langle\varphi \mid \psi\rangle \quad \begin{array}{ll}\text { physical } & \\ = & \text { heaning }\end{array}\right\}$\n$\langle\psi \mid \varphi\rangle^{*}$

\section*{it follows that}
The probability of preparing $|"psi">$ and then measuring $|"phi">$

$$ \frac{\text { IS }}{p}=|\langle\varphi \mid \psi\rangle|^{2} \quad \frac{\text { AKA }}{\text { F/ISEUY }}
$$ 

\begin{center}
\begin{tabular}{ll}
Orthogonal & $\langle\varphi \mid \psi\rangle=0 \quad$ are $\quad$ oisting $\quad$ tashable \\
\end{tabular}
\end{center}

WHY ? B) BECAUSE IT inTRODUCES A CONCEPT OF "STATE AISTANCE"\METRIC" ?

$$ \text { EXAMPLE BURES } D_{B}("psi", \varphi)=\sqrt{2(1-\K \psi|\varphi\rangle)}
$$ 

(1B) Superposition and Interference\
input states $\left|\psi_{1}\right\rangle$ or $\left|\psi_{2}\right\rangle$, heasuring probability of output $|"phi">$
AMPUITUDES $\quad\left\langle\varphi \mid \psi_{1}\right\rangle=c_{1} \quad$ DROBABICITIES $\quad\left|\left\langle\varphi \mid \psi_{1}\right\rangle\right|^{2}=\left|c_{1}\right|^{2}=p_{1}$

$$ \left\langle\varphi \mid \psi_{2}\right\rangle=c_{2} \longrightarrow\left|\left\langle\varphi \mid \psi_{2}\right\rangle\right|^{2}=\left|c_{2}\right|^{2}=p_{2} $$

SOR $\begin{aligned} & \text { ORTHOGONAL } \\ & \text { SIMPLITY }\end{aligned} \quad \begin{aligned} & \left\langle\psi_{1} \mid \psi_{2}\right\rangle=0 \\ & \end{aligned}|+\rangle=\frac{\left|\psi_{1}\right\rangle+\left|\psi_{2}\right\rangle}{\sqrt{2}}$ IS NORMALIZES\
remainder\
\includegraphics[width=0.5\textwidth, center]{2025_10_16_f02af6fa434c9f0bcc00g-04}\
Actually the peobability is

$$ p=\frac{|\langle\psi \mid \varphi\rangle|^{2}}{\langle\psi \mid \psi\rangle\langle\varphi \mid \varphi\rangle} \quad \leqslant \frac{\text { Physical } Q . \Leftrightarrow \begin{array}{l}\text { INVARIANT UNDER } \\ \text { 'GAUGEE TRALFFORM. }\end{array}}{\begin{array}{r}|\psi\rangle \rightarrow \lambda|\psi\rangle \text { WITH } \lambda \in \mathbb{C}(\lambda \neq 0) \\ \text { THUS WE WORK WITH NORMALIZES } \\ \text { GUANTUH STATES }\langle\psi \mid \psi\rangle=\langle\varphi \mid \varphi\rangle=1\end{array}} $$

IC Ortmonormal Bases\
FOR AU PRACTICAL purposes

$$ \operatorname{DIMENSION}(\text { If })=7^{\text {FINITE }}{ }^{\text {COUNTABLE INFINITE }}
$$ 

Why? Because 10. The lab/sample is finite size\
(2.) WE WORK AT BOUNSES EWEREY

DEFINE AN ORTHONORMAL BASIS $\left\lvert\, \begin{gathered}\text { n } \\ \uparrow\end{gathered}\right.$ BAND WIOTH

LABEL = ONE OR MORE INTEGERS\
$\begin{array}{cc}\langle n| n\left\rangle=\delta_{n, n^{\prime}}\right. & \begin{array}{c}\text { KRONECKER } \\ \text { DECTA } \\ \text { NOT SIRAC }\end{array} \\
\text { HIMSENT METRIC } & \end{array} \quad |\psi\rangle=\sum_{\substack{\mid \\ \text { GOES IN HERE }}}} C_{n}|n\rangle \quad \begin{gathered}\text { COMPLETE- } \\ \downarrow \\ \text { NESS }\end{gathered}$

\section*{1D OPERATORS}
they are ensomorphisms of ff (linear and H $\rightarrow$ H )\
NOTATION $\rightarrow \hat{A}|\psi\rangle$
WHERE $\langle\varphi \mid \psi\rangle$

$$ \begin{aligned}
& \left\langle\psi_{1}\right| \hat{A}\left|\psi_{2}\right\rangle \stackrel{\text { def }}{=}\left(\left|\psi_{1}\right\rangle, \hat{A}\left|\psi_{2}\right\rangle\right)_{\text {APAIES TO THE RIGHT }} \\
& =\left(\begin{array}{c}\text { definition of } \\ \text { HERMITIAN } \\ \text { CONGUEATE }\end{array}\right) \quad \left(\hat{A}^{+}\left|\psi_{1}\right\rangle,\left|\psi_{2}\right\rangle\right)=\left(\begin{array}{c}\text { HIL BERY } \\ \text { SCALAR } \\ \text { PRODET } \\ \text { PROPERTY }\end{array}\right) \quad \left(\left|\psi_{2}\right\rangle, \hat{A}^{+}\left|\psi_{1}\right\rangle\right)^{*} \\
& =\left(\left\langle\psi_{2}\right| \hat{A}^{+}\left|\psi_{1}\right\rangle\right)^{*}
\end{aligned} $$

(2.) Observables & heasurement\
an observable is a hecmitian operator $\theta=\theta^{+}$\
OR MORE PRECISELY $\langle\psi| \theta|\varphi\rangle=(|\psi\rangle, \theta|\varphi\rangle)=(\theta|\psi\rangle,\left|\varphi\right\rangle)=\langle\varphi| \theta|\psi\rangle^{*}$

$$ \begin{aligned}
& \theta=\theta^{+} \\
& \text {SPECTRAL } \\ & \text { THEOREM }
\end{aligned}\left\{\begin{array}{ccc}\theta \text { CAN BE DIAGONALIZED } & \\
& \& \\
& \text { ITS EIGENBASIS IS ORTHOGONAL } & \left(\begin{array}{c}\text { AT LEAST ONE } \\ \text { H EIGENGASIS } \\ \text { I SORHOGONAL }\end{array}\right) \\
& \text { EIGENVALUES ARE } & \text { REAL } \
\end{array}\right. $$

$$ G=U D U^{+} \underset{\text { DIAGONAC & REAL }}{U U^{+}=U^{+} U=\mathbb{1}} \quad U=\sum_{\substack{j \\ \sum_{\text {EIGENVALUES }}}}} \eta_{j} \mid J \times\left\langle\varphi_{j}\right| \quad\left(\eta_{1}=\eta_{j}^{*}\right) $$

$2A$ The (hard) measurement process\
\includegraphics[width=0.5\textwidth, center]{2025_10_16_f02af6fa434c9f0bcc00g-06}

$$ \begin{gathered}
\text { For EVERY OUTCOME } \\ \lambda \in \operatorname{SPCC}\{\theta\} \\ \Downarrow \\ \Pi_{\lambda} \quad \begin{array}{c}\text { PROSECTOR OVER } \\ \text { THE EIGENSPACE }\end{array} \\ \theta \Pi_{\lambda}|\phi\rangle=\Pi_{\lambda}|\phi\rangle \lambda \\ {\left[\Pi_{\lambda}, \theta\right]=0} \\ \Pi_{\lambda}=\Pi_{\lambda}^{2}=\Pi_{\lambda}^{\dagger}
\end{gathered} $$

$$ \begin{aligned}
& \left.\| \psi_{\lambda}^{\prime}\right\rangle=\frac{\pi_{\lambda}|\psi\rangle}{\sqrt{\langle\psi| \pi_{\lambda}|\psi\rangle}}
\end{aligned} $$

THIS PROCESS BREAKS TIME REVERSAL (it is FINE BECAUSE it is AN EFFECTIVE PICTURE)

$$ \langle\psi\rangle=\sum_{\lambda}^{1} \lambda p_{\lambda}=\langle\psi|\left(\Sigma \lambda \Pi_{\lambda}\right)|\psi\rangle=\langle\psi| \circlearrowleft|\psi\rangle $$

$$ \Delta \theta^{2}=\left\langle\theta^{2}\right\rangle-\langle\theta\rangle^{2} $$

$=\left\langle\left(\theta-\left.\langle\theta\rangle\right|^{2}\right\rangle

2B operators & observables so not commute

\section*{i[A, B]=C}
$\Delta A \Delta B \geqslant \frac{1}{2}|\langle i[A, B]\rangle| \begin{aligned} & \text { Heisenserg } \\ & \text { uncertanty principle }\end{aligned}$

\section*{in practice}
\begin{center}
\begin{tabular}{|l|l|l|l|l|}
\hline measure A & A 15 DETERMINES & MEASURE B & B IS DETERMINES & A IS NO HORE DETERMINED \\
\hline HEISENBERG IS NOT ALWAYS A HARD BOWND & $\Delta p \Delta x \geqslant$ & $\frac{\hbar}{2}$ & HARD BOUND & $\Delta p \rightarrow 0$ MEANS $\Delta x \rightarrow \infty$ \\
\hline
\end{tabular}
\end{center}

\section*{BUT CONSIDER}
$|0\rangle=\binom{1}{0}$

$$ \sigma^{z}=\binom{1}{-1} \quad \sigma^{x}=\left(\begin{array}{ll}0 & 1 \\ 1 & 0\end{array}\right) \quad \sigma^{y}=\left(\begin{array}{ll}-i & \\ i & \end{array}\right) $$

$|1\rangle=\binom{0}{1}$

$$ \Delta \sigma_{0}^{z} \Delta \sigma_{0}^{x} \geqslant \frac{1}{2}\left|\left\langle\sigma^{y}\right\rangle_{0}\right| $$

$\Delta \sigma_{0}^{z}=\langle 0|\left(\sigma^{z}-\langle 0| \sigma^{z}|0\rangle\right)^{2}|0\rangle=\langle 0|\left(\sigma^{z}-1\right)^{2}|0\rangle=\sqrt{0}\left(\begin{array}{ll}0 & 0 \\ 0 & 4\end{array}\right)\binom{1}{0}=0$
$\Delta \sigma_{0}^{x}=\langle 0|\left(\sigma^{x}-\langle 0| \sigma^{x}|0\rangle\right)^{2}|0\rangle=\langle 0| \sigma^{x^{2}}|0\rangle=\langle 0 \mid 0\rangle=1$
$\left\langle\sigma_{0}^{y}\right\rangle_{0}=\frac{10}{10}\binom{-i}{i}\binom{1}{0}=0$
\includegraphics[width=0.5\textwidth, center]{2025_10_16_f02af6fa434c9f0bcc00g-07}

$$ 0 \geqslant 0 $$

WEU, WHATEVER\
NOT REACY HARD BOUND\
that's one reason why Q-IN.O MAKES SENSE\[0pt]
[2C] Physical transformations of a closes quantum system\
IF A SYSTEM IS CLOSED

$$ \left.\left.|"psi"> \rightarrow\left|\psi^{\prime}\right\rangle={ }^{*} I(|\psi\rangle)=|T| \psi\right\rangle\right
angle $$

(1) IT STAYS DETERMINISTIC\
(2) If PRESERVES TOTAL PROBALIKY

$$ \langle T("psi")| \mathbb{1}|T("psi")\rangle=\langle\psi| \mathbb{1}|\psi\rangle=1 $$

under physical transformations

$$ (T \psi, T \psi)_{H}=(\psi, \psi)_{H} \quad \forall \psi\left[\begin{array}{c}\text { THEREE ARE ONCY TWO } \\ \text { POSSIBICITIES }\end{array}\right. $$

\section*{T ANTI-UNITARY}
$(T \psi, T \varphi)=(\psi, \varphi)^{*} \quad \forall \psi \varphi$
PROBLEM CANNOT CONTINUUJSLY connect with trivital trafo $\mathbb{1}_{.}$ MAOSSIBLE TO ACHIEVE WITH increhental changes

$$ \binom{\text { STIU USEFUL AS A }}{\text { SYMHETRY }}
$$ 

T UNITARY\
$(T \psi, T \varphi)=(\psi, \varphi) \quad \forall \psi, \varphi$
$1_{\text {THIS IS THE COMMON CASE: }}$ CONTINUOUSLY CONNECTES TO 11 SO TIME EVOLUTION OPERATORS ARE OF THIS CCASS

$$ \left|\psi\left(t^{\prime}>t\right)\right\rangle=U\left(t, t^{\prime}\right)|\psi(t)\rangle $$

\begin{itemize}
  \item change of reference frame can be both untary and anti-unitary\
\includegraphics[width=0.5\textwidth, center]{2025_10_16_f02af6fa434c9f0bcc00g-08(1)}\
\includegraphics[width=0.5\textwidth, center]{2025_10_16_f02af6fa434c9f0bcc00g-08(2)}
\end{itemize}

\begin{figure}[h]
\begin{center}
\captionsetup{labelformat=empty}
\caption{hanes sense: stacking hultiple AHYSICA OPS IS STILE A PUYSICAL Op.}
  \includegraphics[width=0.5\textwidth]{2025_10_16_f02af6fa434c9f0bcc00g-08}
\end{center}
\end{figure}

(3) Time Evolution

INITIAC CONDITIONS

CLOSES SYSTEM EVO\
\includegraphics[width=0.5\textwidth, center]{2025_10_16_f02af6fa434c9f0bcc00g-09}

3A. Schrödinger PICTURE - Evolving Stotes

$$ i \hbar \frac{d}{d t}|\psi\rangle=\hat{H}|\psi\rangle \quad \begin{gathered}\text { SCHIÖSINGER'S } \\ \text { EQUATION }\end{gathered} \quad \begin{aligned}
& \text { WHERE } \\ & H=H^{+} \end{aligned} $$

holss for EVERY quantum system evo that is
(a) PHYSICAL
(2) $\operatorname{closes}$
(3) FIRST-ORSER SIFFERENTIAL IN TIME\
(in a way, schrósinger can be relativistic if $H$ is relativistic)

\section*{formal Solutions $H$ is $t$ insepensent}
Ao Diagonalize $H=\sum \varepsilon_{s}\|\varepsilon_{s} \times \varepsilon_{s}\| \quad\binom{\text { so THAT }}{H\left|\varepsilon_{s}\right\rangle=\left|\varepsilon_{s}\right\rangle \varepsilon_{s}}$

$$ \left|\varepsilon_{j}\right\rangle \xrightarrow{E V O C U T I O N} e^{-i \varepsilon_{j} t / \hbar}\left|\varepsilon_{j}\right\rangle $$

B. Expand any intital state $\left|\psi_{0}\right\rangle$ in the eigenbasis

$$ \begin{aligned}
& \left\langle\varepsilon_{j} \mid \psi_{0}\right\rangle=c_{j} \\
& \left|\psi_{0}\right\rangle=\sum_{1} c_{j}\|\varepsilon_{j}\rangle \xrightarrow{\text { EVOLVE }}|\psi(t)\rangle=\sum_{1}^{1}\left|\varepsilon_{j}\right\rangle c_{j} e^{-i \varepsilon_{j} t / \hbar}
\end{aligned} $$

(BANDWIJTH!),\
c. formal expression with the matrix exponential

$$ \begin{aligned}
|\psi(t)\rangle & =\underbrace{\left(\sum_{1}\left|\varepsilon_{j}\right\rangle e^{-i \varepsilon_{j} t / \hbar}\left\langle\varepsilon_{j}\right|\right)\left|\psi_{0}\right\rangle}_{\substack{\text { ANS TAS IS INSEES UNTARY } \\ \text { AND ADATIVE } U\left(t_{2}\right) U\left(t_{1}\right)=U\left(t_{1}+t_{2}\right)}} \\
\exp (-i H(t / \hbar) & \left|\psi_{0}\right\rangle
\end{aligned} $$

DONT FORGET THAT $\exp (A)=1+A+\frac{A^{2}}{2}+\frac{A^{3}}{6}+\ldots=\sum_{j=0}^{\infty} \frac{A^{j}}{j!}$ BUT ALSO $\exp \left(V A V^{+}\right)=V \exp (A) V^{+}$SO ORTENTIMES\
(e) DIAGONALIEE\
(2) examentate the eigenvalues

3B. Heisemberg PICTURE - Evolving OPS

$$ \begin{array}{rlr}
\langle\theta\rangle_{t}=\langle\psi| \theta|\psi\rangle_{t}=\left\langle\psi_{0}\right| \underbrace{+}_{\vec{\theta}(t)}(t) \theta(t)\|\psi_{0}\rangle & \dot{U}(t)=\frac{d}{d t}\left(e^{-i H t / \hbar}\right) \\
\frac{d}{d t} \tilde{\theta}(t)=\dot{U}^{+}(t) \theta U(t)+U^{+}(t) \theta \dot{U}(t) & \left(-\frac{i}{\hbar} H\right) U(t) \\
& =+\frac{i}{\hbar} H U^{+} \theta U-\frac{i}{\hbar} U^{+} \theta U M & {[H, U]=0} \\
\dot{\omega} & =\frac{i}{\hbar}[H, \tilde{\theta}] \quad\left(+U^{+} \dot{\theta} U\right)
\end{array} $$

\section*{3C EXAMPLE: DRIVEN 2-LEVEL SYSTEM AND TRANSITIONS}
\begin{center}
\includegraphics[width=0.5\textwidth]{2025_10_16_f02af6fa434c9f0bcc00g-11}
\end{center}

$$ \begin{aligned}
& \left|\psi_{0}\right\rangle=0 \\
& \left\lvert\, \begin{array}{l}
\mid=\hbar\left(\Omega \sigma^{x}+\Delta \sigma^{z}\right) \\ \binom{\text { RABI }}{\text { REQ. }} \quad \text { (DETUNING) } \leftarrow \text { WE WIU SEE } \\ \text { WHY THIS } \\ \text { IS } \\ \text { THE CASE }\end{array}\right.
\end{aligned} $$

$$ H=\hbar \Omega\left(\begin{array}{ll}0 & 1 \\ 1 & 0\end{array}\right)+\hbar \Delta\left(\begin{array}{cc}1 & 0 \\ 0 & -1\end{array}\right)=\hbar\left(\begin{array}{cc}\Delta & \Omega \\ \Omega & -\Delta\end{array}\right) $$

REWRITE $\ \widetilde{\Omega}=\sqrt{\Omega^{2}+\Delta^{2}} \quad \theta=\arctan \left(\frac{\Delta}{\Omega}\right)$

$$ \begin{aligned}
& \rightarrow \Omega=\Omega \sin \theta \quad \Delta=\tilde{\Omega} \cos \theta \\
& H=\hbar \widetilde{\Omega}(\vec{n} \cdot \overrightarrow{\vec{\sigma}}) \\ & \text { NOTICE THAT }(\vec{n} \cdot \vec{\sigma})^{2}=\mathbb{1}
\end{aligned} \vec{n}=\left(\begin{array}{c}\cos \theta \\ 0 \\ \sin \theta\end{array}\right) \quad \vec{\sigma}=\left(\begin{array}{c}\sigma^{x} \\ \sigma^{y} \\ \sigma^{z}\end{array}\right) $$

Therefore $\exp \left(-\frac{i \mu t}{\hbar}\right)=\sum_{J} \frac{(-i \tilde{\Omega} t)^{j}}{J!}(\vec{n} \cdot \hat{\vec{\sigma}})^{J}$

$$ \begin{aligned}
= & 1 \sum_{j}^{\text {ever }} \frac{(-i \tilde{\Omega} t)}{j !}+(\vec{n} \cdot \hat{\vec{\sigma}}) \sum_{j}^{\cos } \frac{(-i \Omega t)}{J!} \\
= & \mathbb{1} \cos (\tilde{\Omega} t)-i \sin (\tilde{\Omega} t)\left(\begin{array}{cc}\sin \theta & \cos \theta \\ \cos \theta & -\sin \theta\end{array}\right) \\
& \cos (\tilde{\Omega} t)\left(\begin{array}{cc}1 & 0 \\ 0 & 1\end{array}\right)-i \sin (\tilde{\Omega} t)\left(\begin{array}{cc}i \sin \theta & \cos \theta \\ \cos \theta & -i \sin \theta\end{array}\right) \\
& \left|\psi_{0}\right\rangle=0 \quad=\quad \binom{1}{0} \\
& |\psi(t)\rangle=\quad \cos (\tilde{\Omega} t)\binom{1}{0}-i \sin (\tilde{\Omega} t)\left(\begin{array}{c}i \sin \theta \\ \cos \theta\end{array}\right)
\end{aligned} $$

probability of measuring (H)

$$ |\langle 1 \mid \psi(t)\rangle|^{2}=\left|\frac{}{01}\binom{\cos \left(\tilde{\Omega}_{t}\right)-i \sin \left(\tilde{\Omega}_{t}\right) \sin \theta}{-i \sin \left(\tilde{\Omega}_{t}\right) \cos \theta}\right|^{2} $$

$$ =\sin ^{2}(\Omega t) \cos ^{2} \theta $$

\begin{center}
\includegraphics[width=0.5\textwidth]{2025_10_16_f02af6fa434c9f0bcc00g-12}
\end{center}

$$ t_{\text {MAX }}=\(2 J+1) \frac{\pi}{2 \dot{\Omega}}=(2 J+1) \frac{\pi}{2 \sqrt{\Omega^{2}+\Delta^{2}}} \quad T=\frac{\pi}{\Omega} $$

$\xrightarrow[\substack{\text {TRANSIGION } \\ \text { PROB. }}]{\text { MAX }} P_{\text {MAX }}=\cos ^{2} \theta=\frac{\Omega^{2}}{\Omega^{2}+\Delta^{2}}=\left(1-\frac{\Delta^{2}}{\Omega^{2}+\Delta^{2}}\right) $
$\Downarrow$
qualitative lesson:\nTo have high transition\
YOU NEES $\triangle \ll \Omega$
(mportant later)
